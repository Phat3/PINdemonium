\chapter{Implementation details}
\label{chapter4}
\thispagestyle{empty}

\section{System architecture}

\section{System details}
In this section we are going to focus in detail about the implementation of each of the six points listed in the previous chapter:
\begin{enumerate}
\item The techniques used in order to retrieve the real EIP exploit some assembly instructions which when are executed need to save the execution context, including the instruction pointer value, in order to work correcly. Some of these are the instructions that trigger exceptions (E.g. INT 2E) or the instructions that save the floating point unit state (E.g. FSAVE). In order to deal with these thecniques we performed a single instruction check, and if we found one of these instruction, after its execution, we restore the context with a fake EIP value (The one that the program would have seen if )
\item
\item the eXait library uses the function VirtuaQuery to retrieve information about the page permissions. In order to avoid this we hooked the VirtualQuery and when the program queries pages outside the whitelist we fake the results and return that those pages are not mapped. Moreover, the VirtualQueryEx has the same function of the VirtualQuery but it allows to specify a handle of the process whose pages we want to query. However, if the handle corresponds to the current process, the VirtuaQueryEx does exactly the same things as the VirtualQuery. In order to prevent this, we patched also the VirtualQueryEx.
\item
\item
\item
\end{enumerate}
